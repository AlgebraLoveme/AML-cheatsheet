\section{PAC Learning}

\begin{itemize}[itemsep=0pt,topsep=0pt, leftmargin=2pt, itemindent=5pt, labelwidth=5pt]
    \item A learning algorithm $\mathcal{A}$ can learn $c\in C$ if there is a poly(.,.), s.t. for (1) any distribution $\mathcal{D}$ on $\mathcal{X}$ and (2) $\forall \epsilon\in [0, 1/2],\delta\in [0, 1/2]$, $\mathcal{A}$ outputs $\hat{c}\in \mathcal{H}$ given a sample of size at least poly($\frac{1}{\epsilon}$, $\frac{1}{\delta}$, size($c$)) such that $P(\mathcal{R}(\hat{c})-\inf_{c\in C}\mathcal{R}(c)\le\epsilon) \ge 1-\delta$.
    \item $\mathcal{A}$ is called an efficient PAC algorithm if it runs in polynomial of $\frac{1}{\epsilon}$ and $\frac{1}{\delta}$.
    \item $\mathcal{C}$ is (efficiently) PAC-learnable from $\mathcal{H}$ if there is an algorithm $\mathcal{A}$ that (efficiently) learns $C$ from $\mathcal{H}$.
    \item Finite $\mathcal{C}$, $\mathbf{P}\left(\mathcal{R}\left(\hat{c}_{n}^{*}\right)-\inf _{c \in \mathcal{C}} \mathcal{R}(c)>\epsilon\right) \leq 2|\mathcal{C}| \exp \left(-\frac{n \epsilon^{2}}{2}\right)$  is PAC-learnable.
    \item $\mathcal{C}$ with $\mathrm{dim}_{VC} = d<\infty$ is PAC-learnable, $\mathbf{P}\left(\mathcal{R}\left(\hat{c}_{n}^{*}\right)-\inf _{c \in \mathcal{C}} \mathcal{R}(c)>\epsilon\right) \leq 9 n^{d} \exp \left(-\frac{n \epsilon^{2}}{32}\right)$
\end{itemize}
